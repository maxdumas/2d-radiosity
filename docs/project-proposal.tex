% hello.text - Our first LaTeX example! :D

%%%%%%%% PREAMBLE %%%%%%%%

\documentclass{article}
\usepackage{times, amssymb, amsmath, amsfonts}
\newcommand{\vect}[1]{\boldsymbol{#1}}

%%%%%%%% END PREAMBLE %%%%%%%%

\begin{document}

\title{Numerical Computing 2014 Project Proposal}
\author{Max Dumas}
\maketitle

\begin{abstract}
For my project I intend to develop a two-dimensional radiosity-based illumination algorithm. The goal of this project is to be able to take a generalized two-dimensional scene, with any sort of polygonal geometry contained within, and create a fully illuminated scene in a timely fashion, with reasonably accurate simulated diffusion of light throughout the open space of the scene. The final scene will, as a result, feature realistically gradated lighting, blocked and reflected by obstacles, resulting in soft, believable shadows. A hopeful stretch goal of this project is to have a solution that works in real-time---that is, the solution will be robust enough and fast enough to be capable of recalculating lighting on-the-fly with changing geometry and light-source properties.
\end{abstract}
\section{Implementation}
Implementation of the project will involve the following:
\begin{enumerate}
\item Creating structures to represent the geometry of the level in such a way that radiosity can be computed efficiently.
\item Computing the radiosity for the geometry of the scene and illuminating the space of the scene.
\item Displaying the illuminated scene.
\end{enumerate}

\section{Scene Representation}
The scene, which will be viewed from directly above, will be primarily represented by two data structures: The space and boundaries of the scene, which will have set, square dimensions, will be discretized into an $m$ by $m$ grid of spaces referred to as $L$, which will store information regarding their position, illumination as a scalar factor between zero and one, and their color. All grid spaces will be square and have a uniform side-length. A standard undirected graph, $G = (V, E)$, will represent the geometry contained within the scene, where $V$ is the set of vertices and $E$ is the set of edges which compose the graph. $E$ will have magnitude $n$. Edges are right-facing, meaning that for an edge $(v, w)$, the normal of the edge points to the right when facing towards vertex $w$ from vertex $v$.

$E$ will also have associated vectors of magnitude $n$ for the following: $\vect{\rho}$ represents the reflectivity of the edges, $\vect{\epsilon}$ represents the emissivity of the edges, and finally $\vect{β}$ represents the radiosity of the edges. The vectors are structured so that any edge $e_i$ in $E$ has reflectivity $\rho_i$, emissivity $\epsilon_i$, and radiosity $\beta_i$.

$G$, $\vect{\rho}$ and $\vect{\epsilon}$ are predefined. $\vect{\beta}$ is our unknown. The bounding dimension of $L$, in terms of grid-spaces, and the width of each grid-space are both also predefined and $L$ is generated at runtime, with each grid-space being assigned a color, a position based on its index in $L$, and its illumination set to zero.

Finally, an $n$ by $n$ matrix $\vect{F}$ will represent the “view-factor” between any given pair of edges, with $0\leq F_{ij}\leq1$ being a scalar representing how visible the “face” of the edge $i$ (as indicated by the direction of its normal) is to the “face” of an edge $j$. This factor will be calculated using the following equation:
\begin{equation}
F_{ij}=\frac{K}{r^2}\cos(\theta_i)\cos(\theta_j)Vis(i, j)
\end{equation}
$\theta_i$ and $\theta_j$ represent the angles between the line segment drawn from the centers of the two edges and the normals of edge $i$ and $j$, respectively. $r$ is the distance between the centers of the edges, and $K$ is an arbitrary constant. $Vis(i, j)$ represents the approximate percentage of $j$ that is accessible by $i$, meaning that it is not blocked by any other geometry. If $i$ and $j$ have no obstacles in between them, $Vis(i, j)$ returns 1. If $j$ is fully occluded by other geometry from $i$, $Vis(i, j)$ returns 0. For partial occlusion, $0<Vis(i,j)<1$.

\section{Computing Radiosity and Illuminating the Scene}
The following equation, given our previously defined data structures, represents our primary radiosity equation:
\begin{equation}
(\vect{I}-\vect{\rho} \vect{F})\vect{\beta}=\vect{\epsilon}
\end{equation}
Where $\vect{\rho}\vect{F}$ is a row-wise multiplication, so each $\rho_i$ is multiplied to the row $F_i$. We can solve for this equation iteratively using the Jacobi or Gauss-Seidel methods, giving us our radiosity with an accuracy of as many iterations as we desire. These methods are guaranteed to converge (and within fairly few iterations) as $(\vect{I}-\vect{\rho}\vect{F})$ is diagonally dominant.

After solving for $\vect{\beta}$, we have radiosity for every edge. Thus, every edge of our geometry has, to some degree, become a light source. However the “surfaces” of the edges are not visible due to our two-dimensional perspective, so it becomes necessary to propagate the light energy that is now “stored” in the edges of the geometry to the grid spaces which comprise the scene. The way I am considering doing this is by using a ray-sweep algorithtm to determine the visibility of each grid space to each edge, creating a new matrix $\vect{H}$ that will be $m$ by $n$, and contain a view factor for each edge to the set of all grid spaces. Light is then propagated outwards from the edges onto the grid spaces by a simple inverse function. For an edge $E_i$ of $E$, the light given to any given grid space $L_{j}$ in $L$ is represented by the following equation:
\begin{equation}
L_{j}=\frac{H_{ij}}{r^2}\beta_i
\end{equation}

\section{Displaying the Illuminated Scene}
For display, I intend to use a simple low-level graphics library such as SDL for C, or a similar library in another language. Grid spaces will displayed as untextured, vertex-colored quads, and edges will simply be displayed as lines. All visible illumination effects will be achieved via the coloring of the grid spaces.
\end{document}